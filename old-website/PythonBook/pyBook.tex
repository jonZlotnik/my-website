\documentclass{book}

\usepackage{listings}

\title{pyBook}
\author{JonnyZ}
\date{\today}

\begin{document}

\maketitle

\tableofcontents




\chapter{pyBasics}

\section{Hello World}
The Hello World program is very simple, it only needs two lines.
The first line begins with a `shebang' (the symbol \#! - also known as a hashbang)
followed by the path to the Python interpreter.
The program loader uses this line to work out what the rest of the lines need to be interpreted with.
If you're running this in an IDE like IDLE, you don't necessarily need to do this.

The code that is actually read by the Python interpreter is only a single line.
We're passing the value Hello World to the print function by playing it in brackets immediately after we've called the print function.
Hello World is enclosed in quotation marks to indicate that it is a literal value and should not be interpreted as source code.
As expected, the print function in Python prints any value that gets passed to it from the console.

\begin{lstlisting}
#!/usr/bin/env python2
print(``Hello World'')
\end{lstlisting}

You can run the Hello World program by prefixing its filename with ./

\section{Variables and data types}
A variable is a name in source code that is associated with an area in memory that you can use to store data, 
which is then called upon throughout the code. 
The data can be one of many types including:

\begin{table}{l|l}

Integer & Stores whole numbers\\\hline
Float & Stores decimal numbers\\\hline
Boolean & Can have a value of True or False\\\hline
String & Stores a collection of characters. ``Hello World'' is a string\\

\end{table}

As well as these main data types, there are sequence types (technically, a string is a sequence type but is so commonly used we've classed iit as a main data type):

\begin{table}{l|l}

List & Contains a collection of data in a specific order\\\hline
Tuple & Contains a collection of immutable data in a specific order.\\

\end{table}

\end{document}